\section{GSM - Gleichstrommotoren}
\vspace{-0.5cm}
\begin{minipage}[b]{0.45\textwidth}
	\centering
	\includegraphics[width=8cm]{images/GSM_Aufbau.png}
\end{minipage}
\begin{minipage}[b]{0.25\textwidth}
	\centering
	\includegraphics[width=5cm]{images/Grundgleichungen.png}
	\vspace{-1cm}
\end{minipage}
\begin{minipage}[b]{0.33\textwidth}
	\vspace{-2cm}
	\centering
	\includegraphics[width=5cm]{images/Ankerrueckwirkung.png}
	\vspace{0.2cm}
\end{minipage}
\vspace{-1cm}
\subsection{Fremderregte GSM}
\vspace{-0.2cm}
\begin{minipage}[b]{0.5\textwidth}
	\raggedright
	\includegraphics[width=6cm]{images/Ersatzschaltbild_GSM.png}
\end{minipage}
\begin{minipage}[b]{0.5\textwidth}
	\raggedright
	\includegraphics[scale = 0.7]{images/KennlinieFremderregt}
\end{minipage}\\
%\begin{minipage}[h]{0.5\textwidth}
%	\subsection{Drehmoment}
%	\begin{tabular}{| P{3cm}|P{3cm}|}
%		\hline
%		sdsd & sdsd \\
%		\hline
%	\end{tabular}
%\end{minipage}
\\
\renewcommand{\arraystretch}{2.5}
\begin{tabular}[b]{| C{4cm} | P{7cm} | P{7cm} |}
\hline
\textbf{Erregerwicklung}	& $U_e = R_e\cdot I_e + L_e\cdot\dfrac{dI_e}{dt}$ & Spannungsgleichung des Statorkreises\\
\hline
\textbf{Ankerwicklung}	& $U_a = R_a \cdot I_a + L_a \cdot \dfrac{dI_a}{dt} + E$ & $E = \omega\cdot\psi$ \qquad $\psi = L_e\cdot I_e \, \widehat{=}$ Erregerfluss \newline \newline $\omega = 2\pi\cdot n$ \newline \quad n $\widehat{=}$ Drehzahl des Läufers $\left[\dfrac{1}{s}\right]$\newline \newline Spannungsgleichung des Rotorkreises\\
\hline
\textbf{Elektrische Leistung} & Im stationären Betrieb: \quad $\left(\dfrac{d}{dt} = 0\right)$ \newline \newline $P_{el} = P_e + P_a = U_e\cdot I_e + U_a\cdot I_a$ \newline \newline $P_{el} = \underbrace{R_e\cdot I_e^2}_{\substack{Ohmsche \\ Erregerverluste}} + \underbrace{R_a\cdot I_a^2}_{\substack{Ohmsche \\ Ankerverluste}} + \underbrace{\omega\cdot\psi\cdot I_a}_{\substack{Mechanische\\Leistung}}$ \newline & $[P] = W$ \\
\hline
\textbf{Mechanische Leistung} & $P_{mech} = \omega\cdot M = \omega\cdot\psi\cdot I_a\, = \omega\cdot L_e \cdot I_e \cdot I_a$ & \\
\hline
\textbf{Drehmoment} & $M = \psi\cdot I_a\, = L_e\cdot I_e\cdot I_a$ & $[M] = Nm$ \\
	\lasthline
\end{tabular}
\clearpage
\newpage
\subsection{Nebenschluss GSM}
Die Erreger- und Ankerwicklung werden parallel an die gleiche Spannungsquelle geschaltet.\newline Beim Nebenschluss sind die Anker- und Erregerspannung gleich und der Anker- und Erregerstrom unabhängig \newline voneinander. \\
\begin{minipage}[b]{0.4\textwidth}
	\raggedright
	\includegraphics[width=6cm]{images/Nebenschluss_GSM.png}
\end{minipage}
\begin{minipage}[b]{0.5\textwidth}
	\raggedright
	\includegraphics[scale = 0.6]{images/KennlinieNebenschluss}
\end{minipage}\\

\renewcommand{\arraystretch}{2.5}
\begin{tabular}{| C{4cm} | P{7cm} | P{7cm} |}
	\firsthline
	\textbf{Drehmoment}	& $U = U_a = R_a\cdot I_a + \omega\cdot\psi$ \newline \newline $I_a = \dfrac{U - \omega\cdot\psi}{R_a}$ \newline \newline $M = I_a\cdot\psi\, = \, I_a \cdot L_e \cdot I_e $ \newline\newline $M = \dfrac{U\cdot\psi}{R_a}-\dfrac{\omega\cdot\psi^2}{R_a}$ \newline & Spannungsgleichung der Nebenschluss-Schaltung\newline \newline $\psi = L_e\cdot I_e \, \widehat{=}$ Erregerfluss \newline \newline $\omega\, = \, 2\pi\cdot n \quad \left[\dfrac{1}{s}\right]$ \\
	\hline
	\textbf{Anlaufmoment} \newline \newline $(n = 0)$	& $M_A = \dfrac{U\cdot\psi}{R_a + R_v}$ \newline \newline $I_{Anlauf} \,= \, \dfrac{U}{R_a + R_v} $ & $[M] = Nm$ \newline $R_a \, \widehat{=}$\, Ankerwiderstand \newline $R_v \, \widehat{=}$ Im Ankerkreis in Serie geschalteter Regelungswiderstand (= \textbf{oft 0})\\
	\hline
	\textbf{Elektrische Leistung} & Im stationären Betrieb: \quad $\left(\dfrac{d}{dt} = 0\right)$ \newline \newline $P_{el} = P_e + P_a = U_e\cdot I_e + U_a\cdot I_a$ \newline \newline $P_{el} = \underbrace{R_e\cdot I_e^2}_{\substack{Ohmsche \\ Erregerverluste}} + \underbrace{R_a\cdot I_a^2}_{\substack{Ohmsche \\ Ankerverluste}} + \underbrace{\omega\cdot\psi\cdot I_a}_{\substack{Mechanische\\Leistung}}$ \newline & $[P] = W$ \\
	\hline
	\textbf{Mechanische Leistung} & $P_{mech} = \omega\cdot M = \omega\cdot\psi\cdot I_a$ & \\
	\hline
	\textbf{Leerlaufdrehzahl}& $n_0 = \dfrac{U}{2\pi\cdot\psi}$  \qquad $(M = 0)$& $[n] = \dfrac{1}{s}$\\
	\lasthline
\end{tabular}
\newpage
\subsection{Reihenschluss GSM}
Die Erreger- und Ankerwicklung werden in Serie an die gemeinsame Spannungsquelle geschaltet.\newline Beim Reihenschluss sind die Anker- und Erregerströme gleich und die Anker- und Erregerspannungen sind deshalb \newline stark voneinander abhängig.\newline
\begin{minipage}[b]{0.4\textwidth}
	\raggedright
	\includegraphics[width=6cm]{images/Reihenschluss.png}
\end{minipage}
\begin{minipage}[b]{0.5\textwidth}
	\raggedright
	\includegraphics[scale = 0.6]{images/KennlinieReihenschluss}
\end{minipage}\\
\begin{minipage}[b]{0.7\linewidth}
\raggedleft
\textcolor{red}{Achtung: M $\rightarrow$ 0 => n $\rightarrow \infty$}
\end{minipage}
\\

\renewcommand{\arraystretch}{2.5}
\begin{tabular}{| C{4cm} | P{7cm} | P{7cm} |}
	\firsthline
	\textbf{Drehmoment}	& $U = (R_a + R_e)\cdot I + 2\pi n\cdot\psi$ \newline \newline $M = I\cdot\psi$ \newline \newline $M = I\cdot\psi = L_e\cdot\left(\dfrac{U}{R_a + R_e + 2\pi n\cdot\psi}\right)^2$\newline \newline $\psi = L_e\cdot I$ & Spannungsgleichung der Reihenschluss-Schaltung \newline \newline $I\,=\,I_a\,=\,I_e$  \\
	\hline
	\textbf{Anlaufmoment}	& $M_A = \dfrac{L_e\cdot U^2}{\left(R_a + R_e\right)^2}$ \qquad $\left(n = 0\right)$ & $[M] = Nm$ \newline \\
	\hline
	\textbf{Bezugsdrehzahl}& $n_b = \dfrac{R_a + R_e}{2\pi\cdot L_e}$ \newline & $[n] = \dfrac{1}{s}$ \\
	\lasthline
\end{tabular}

\subsection{Drehzahlregelung}
\begin{minipage}[b]{0.33\textwidth}
	\raggedright
	\includegraphics[width=5cm]{images/Widerstandsregelung.png}
\end{minipage}
\begin{minipage}[b]{0.33\textwidth}
	\raggedright
	\includegraphics[width=5cm]{images/Spannungsregelung.png}
\end{minipage}
\begin{minipage}[b]{0.33\textwidth}
	\raggedright
	\includegraphics[width=5cm]{images/Erregerstromregelung.png}
\end{minipage}






