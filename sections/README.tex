\thispagestyle{empty}
\setcounter{page}{0} %Set PageNumber to 0
{\huge README }
\section*{Beschreibung}
Formelsammlung für Elektrische Maschinen auf Grundlage der Vorlesung FS 16 von Prof. Dr.Jasmin Smajic \newline
Bei Korrekturen oder Ergänzungen wendet euch an einen der Mitwirkenden.
OKEY?

\section*{Modulschlussprüfung}
Prüfungsstoff ist der gesamte ElMasch-Vorlesungsinhalt des FS2016 einschliesslich aller UE + P Lerninhalte.\newline
Als Hilfsmittel für die Modulschlussprüfung sind die Vorlesungen,\newline
UE-Aufgaben und eigenen Praktikumsaufzeichnungen sowie der Taschenrechner erlaubt.

\subsection*{Plan und Lerninhalte}
{\scriptsize 
\begin{itemize}
    \item Die elektrodynamischen Grundgesetze 
    \item Allgemeine Merkmale elektrischer Maschinen 
    \item Gleichstrommaschinen: 
    \subitem Aufbau und Wirkungsweise, Grundgleichungen, Betriebsverhalten, Klemmenbezeichnungen und Schaltungen 
    \item Synchronmaschinen: Bauarten, Funktion, Betriebseigenschaften - und grössen, Sonderbauformen 
    \item Drehstrom-Asynchronmaschinen: 
    \subitem Bauarten, Schaltungen, Wirkungsweise, Betriebseigenschaften, Kreisdiagramm 
    \item Transversalflussmaschinen: Aufbau und Wirkungsweise, Betriebseigenschaften
    \item Linearantriebe
    \item Steuerung und Regelung elektrischer Maschinen
\end{itemize}
}
\vfill
\section*{Contributors}
\begin{tabular}{ll}
	Stefan Reinli   & stefan.reinli@hsr.ch   \\
	Luca Mazzoleni  & luca.mazzoleni@hsr.ch  \\
	Michel Gisler   & michel.gisler@hsr.ch   \\
	Severin Kundert & severin.kundert@hsr.ch
\end{tabular} 

{\scriptsize 
\section*{License}
\textbf{Creative Commons BY-NC-SA 3.0}

Sie dürfen:
\begin{itemize}
    \item Das Werk bzw. den Inhalt vervielfältigen, verbreiten und öffentlich
    zugänglich machen.
    \item Abwandlungen und Bearbeitungen des Werkes bzw. Inhaltes anfertigen.
\end{itemize}
Zu den folgenden Bedingungen:
\begin{itemize}
    \item Namensnennung: Sie müssen den Namen des Autors/Rechteinhabers in der von ihm
    festgelegten Weise nennen.
    \item Keine kommerzielle Nutzung: Dieses Werk bzw. dieser Inhalt darf nicht für
    kommerzielle Zwecke verwendet werden.
    \item  Weitergabe unter gleichen Bedingungen: Wenn Sie das lizenzierte Werk bzw. den
    lizenzierten Inhalt bearbeiten oder in anderer Weise erkennbar als Grundlage
    für eigenes Schaffen verwenden, dürfen Sie die daraufhin neu entstandenen
    Werke bzw. Inhalte nur unter Verwendung von Lizenzbedingungen weitergeben,
    die mit denen dieses Lizenzvertrages identisch oder vergleichbar sind.
\end{itemize}
Weitere Details: http://creativecommons.org/licenses/by-nc-sa/3.0/ch/
}
%If we meet some day, 
%and you think this stuff is worth it, you can buy me a beer in return.
\clearpage
\pagenumbering{arabic}% Arabic page numbers (and reset to 1)