\section{Asynchronmotor}
\textcolor{green}{Vorteil}:
\begin{itemize}
	\item sehr einfacher Aufbau
	\item sehr robust und widerstandsfähig 
\end{itemize}
\textcolor{red}{Nachteil}:
\begin{itemize}
	\item Extrem hoher Anlaufstrom \newline
		$\Rightarrow$ dies wird vermindert mit der Stern-Dreieck-Umschalt-Methode \newline
        (Im Dreieck 3 Mal mehr Leistung im Dreieckbetrieb)
\end{itemize}

\subsection{Aufbau  Ständer}
    \begin{minipage}[b]{0.45\linewidth}
    	\includegraphics[scale = 0.3]{images/AsynchronmotorAufbau}
    \end{minipage}
    \begin{minipage}[b]{0.28\linewidth}
    	\includegraphics[scale = 0.3]{images/AQuerschnitt}
    \end{minipage}
    \begin{minipage}[b]{0.33\linewidth}
    	\includegraphics[scale = 0.4]{images/AsynchronmotorStaenderkern}
    \end{minipage}\\
    \subsection{Aufbau Läufer}
    \begin{minipage}[b]{0.5\linewidth}
    	\includegraphics[scale = 0.4]{images/AsynchronRotor}
    \end{minipage}
    \begin{minipage}[b]{0.5\linewidth}
    	\includegraphics[scale = 0.4]{images/QuerschnittAsynchronrotor}
    \end{minipage}
    \\ \\
    Wirbelströme im Eisen entstehen durch die Induktion des Rotors in den Stator \newline
    $\Rightarrow$ Stator erwärmt sich.\newline
    Durch den Rillenaufbau des Stators können diese Wirbelströme bzw. die Temperaturansteigung minimiert werden.
    \\
    Schlupf $\widehat{=}$ der Abweichung zu der Synchronen Drehzahl 
    \clearpage
    \pagebreak

\subsection{Formeln Läufer}
    \includegraphics[scale = 0.4]{images/QuerschnittAmotor}
    \\ \\
    \begin{longtable}{| p{.25\textwidth} | p{.40\textwidth} | p{.30\textwidth} |}
    	\hline
    	\textbf{Induzierte Spannung} &
        \[ U_i = 4.44\cdot f\cdot w\cdot\xi\cdot\phi \] &
        f $\widehat{=}$ Frequenz \newline
        w $\widehat{=}$ Windungszahl \newline
        $\xi$ = Wicklungsfaktor \newline
        $\phi$ = Magnetischer Fluss
        \\ \hline
        
        \textbf{Elektromagnetische Kraft}	&
        \begin{equation*} \vec{F_2} = I_2\cdot\vec{I_2}\times\vec{B_1}\end{equation*} &
        \\ \hline
        
        \textbf{Mech. Drehmoment}	&
        \begin{equation*}\vec{M_2} = \vec{r_2}\times\vec{F_2}\end{equation*}&
        \\ \hline
        
        \textbf{Relativdrehzahl}&
        \[ n_1= \frac{f_1}{p}\]
        \[ n_2=n_1 - n \]&
        n = Drehzahl des Läufers \newline
        $n_1$ = Synchrondrehzahl \newline
        $ n_2 $ = Relativdrehzahl
        \\ \hline
        
        \textbf{Schlupf}&
        \[ s= \frac{n_2}{n_1}=\frac{n_1-n}{n_1}=\frac{f_2}{f_1} \]&
        $ f_1 $ = Frequenz Drehfeld \newline
        $ f_2 $ = Frequenz Anker \newline
        Synchroner Lauf: s = 0 \newline
        Stillstand: s = 1
        \\ \hline 
        
        \textbf{Induzierter Strom des Läufers}&
         \[ I_2 = \frac{U_{i20}}{\sqrt{R_2^2+X_{2\sigma^2}}} \]&
         \\ \hline
        
        \textbf{Stillstand}&
        s = 1 \newline
        $ f_2 = f_1 $ \newline
        \[ I_2 = I_{2max} = \frac{U_{i20}}{\sqrt{(R_2/2)^2+X_{2\sigma^2}}} \]&
         \newline
        \tabbild[scale=0.3]{images/FlussStillstand}       
        \\ \hline
        
         \textbf{Synchronlauf}&
         s = 0 \newline
         $ f_2 = 0 $\newline
         \[ I_2 = I_{2max} = 0\]&
          \newline
         \tabbild[scale = 0.3]{images/FlussSynchron}
         \\ \hline

        
        \textbf{Verluste}\newline
        \tabbild[scale = 0.3]{images/PVerluste}&
        \[ P_{D1}=P_m+P_{C22} \]
        \[ P_{D1}=2\cdot\pi\cdot n_1\cdot M \quad
         P_m = 2\cdot\pi\cdot n\cdot M \]
        \[ M = \frac{1}{2 \pi n_1}\frac{P_Cu2}{s} \]&
         $ P_1 $ - primäre Netzleistung \newline
         $ P_{Cu} $ - Ohmsche Verluste \newline
         $ P_{Fe} $ - Blechkernverluste \newline
         $ P_{D1} $ - Drehfeldleistunf \newline
         $ P_m $ - mechanische Leistung \newline
         $ P_r $ - Reibungsverluste und Lüftung \newline
         $ P_m ' $ - mech. Nutzleistung \newline
         M - Drehmoment
        \\ \hline
        
        
    \end{longtable}
    \clearpage
    \pagebreak
