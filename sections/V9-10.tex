\section{Asynchronmotor}
Eigenschaften:
\begin{itemize}
    \item Meist verwendete Elektromotor
    \item Drehfekd wird durch den Ständer erzeugt
    \item Drehmoment entsteht durch den im Läufer induzierten Strom
\end{itemize}
\textcolor{green}{Vorteil}:
\begin{itemize}
	\item sehr einfacher Aufbau
	\item sehr robust und widerstandsfähig 
\end{itemize}
\textcolor{red}{Nachteil}:
\begin{itemize}
	\item Extrem hoher Anlaufstrom \newline
		$\Rightarrow$ dies wird vermindert mit der Stern-Dreieck-Umschalt-Methode \newline
        (Im Dreieck 3 Mal mehr Leistung im Dreieckbetrieb)
\end{itemize}

\subsection{Aufbau  Ständer}
    \begin{minipage}[b]{0.45\linewidth}
    	\includegraphics[scale = 0.3]{images/AsynchronmotorAufbau}
    \end{minipage}
    \begin{minipage}[b]{0.28\linewidth}
    	\includegraphics[scale = 0.3]{images/AQuerschnitt}
    \end{minipage}
    \begin{minipage}[b]{0.33\linewidth}
    	\includegraphics[scale = 0.4]{images/AsynchronmotorStaenderkern}
    \end{minipage}\\
    \subsection{Aufbau Läufer}
    \begin{minipage}[b]{0.5\linewidth}
    	\includegraphics[scale = 0.4]{images/AsynchronRotor}
    \end{minipage}
    \begin{minipage}[b]{0.5\linewidth}
    	\includegraphics[scale = 0.4]{images/QuerschnittAsynchronrotor}
    \end{minipage}
    \\ \\
    Wirbelströme im Eisen entstehen durch die Induktion des Rotors in den Stator \newline
    $\Rightarrow$ Stator erwärmt sich.\newline
    Durch den Rillenaufbau des Stators können diese Wirbelströme bzw. die Temperaturansteigung minimiert werden.
    \\
    Schlupf $\widehat{=}$ der Abweichung zu der Synchronen Drehzahl 
    \clearpage
    \pagebreak

\subsection{Formeln Läufer}
    \includegraphics[scale = 0.4]{images/QuerschnittAmotor}
    \\ \\
    \begin{longtable}{| p{.25\textwidth} | p{.40\textwidth} | p{.30\textwidth} |}
    	\hline
    	\textbf{Induzierte Spannung} &
        \[ U_i = 4.44\cdot f\cdot w\cdot\xi\cdot\phi \] &
        f $\widehat{=}$ Frequenz \newline
        w $\widehat{=}$ Windungszahl \newline
        $\xi$ = Wicklungsfaktor \newline
        $\phi$ = Magnetischer Fluss
        \\ \hline
        
        \textbf{Elektromagnetische Kraft}	&
        \begin{equation*} \vec{F_2} = I_2\cdot\vec{I_2}\times\vec{B_1}\end{equation*} &
        \\ \hline
        
        \textbf{Mech. Drehmoment}	&
        \begin{equation*}\vec{M_2} = \vec{r_2}\times\vec{F_2}\end{equation*}&
        \\ \hline
        
        \textbf{Relativdrehzahl}&
        \[ n_1= \frac{f_1}{p}\]
        \[ n_2=n_1 - n \]&
        n = Drehzahl des Läufers \newline
        $n_1$ = Synchrondrehzahl \newline
        $ n_2 $ = Relativdrehzahl
        \\ \hline
        
        \textbf{Schlupf}&
        \[ s= \frac{n_2}{n_1}=\frac{n_1-n}{n_1}=\frac{f_2}{f_1} \]&
        $ f_1 $ = Frequenz Drehfeld \newline
        $ f_2 $ = Frequenz Anker \newline
        Synchroner Lauf: s = 0 \newline
        Stillstand: s = 1
        \\ \hline 
        
        \textbf{Induzierter Strom des Läufers}&
         \[ I_2 = \frac{U_{i20}}{\sqrt{(R_2/s)^2+X_{2\sigma}^2}} \]&
         \\ \hline
        
        \textbf{Stillstand}\newline
         s = 1 \newline
        $ f_2 = f_1 $ &
        \[ I_2 = I_{2max} = \frac{U_{i20}}{\sqrt{R_2^2+X_{2\sigma}^2}} \]&
         \newline
        \tabbild[scale=0.3]{images/FlussStillstand}       
        \\ \hline
        
         \textbf{Synchronlauf} \newline
          s = 0 \newline
         $ f_2 = 0 $&
         \[ I_2 = I_{2max} = 0\]&
          \newline
         \tabbild[scale = 0.3]{images/FlussSynchron}
         \\ \hline

        
        \textbf{Verluste Drehmoment}\newline
        \tabbild[scale = 0.3]{images/PVerluste}&
        \[ P_{D1}=P_m+P_{C22} \]
        \[ P_{D1}=2\cdot\pi\cdot n_1\cdot M \quad
         P_m = 2\cdot\pi\cdot n\cdot M \]
        \[ M = \frac{1}{2 \pi n_1}\frac{P_{Cu2}}{s} \]&
         $ P_1 $ - primäre Netzleistung \newline
         $ P_{Cu} $ - Ohmsche Verluste \newline
         $ P_{Fe} $ - Blechkernverluste \newline
         $ P_{D1} $ - Drehfeldleistunf \newline
         $ P_m $ - mechanische Leistung \newline
         $ P_r $ - Reibungsverluste und Lüftung \newline
         $ P_m ' $ - mech. Nutzleistung \newline
         M - Drehmoment
        \\ \hline
        
        \textbf{Funktion Drehmoment} \newline
        \tabbild[scale = 0.4]{images/FunktionDrehmoment}&
        \[ M=\frac{1}{2\pi n_1}\frac{\textcolor{blue}{P_{Cu2}}}{s} \]
        \[= \frac{\textcolor{blue}{q_2}}{2\pi n_1}\cdot \textcolor{green}{I_2^2}\cdot\frac{\textcolor{blue}{R_2}}{s} \]
        \[= \frac{q_2}{2\pi n_1}\textcolor{green}{\frac{U_{i20}^2}{(R_2/s)^2+X_{2\sigma}^2}}\frac{R_2}{s} \]&
        \textcolor{blue}{M-Kennlinie} \newline
        \textcolor{yellow}{Motorbetrieb} \newline
        \textcolor{green}{Generator-Betrieb}
        \\ \hline
        
        \textbf{Anlauf} \newline
        \tabbild[scale=0.4]{images/ASMAnlauf}&
        \[ M=\frac{q_1}{2\pi n_1}\cdot \frac{U_1^2}{(R_2'/s)^2+X_{2\sigma}^2}\cdot\frac{R_2'}{s} \]
        \[ s_K=\frac{R_2'}{X_{2\sigma}'} \]
        \[ M_K= \frac{q_1}{4\pi n_1}\cdot\frac{U_1^2}{X'_{2\sigma}} \]&
        $ q_1 $= Anz Phasen der Statorwiklung\newline
        \\ \hline
        
        \textbf{Klosssche Gleichung}&
        \[ \frac{M}{M_K}=\frac{2}{\frac{s}{s_K}+\frac{s_K}{s}} \]
        \[ \frac{1}{2} sk^2 - \frac{2 Mk}{M}sk+s=0 \]&
        \\ \hline        
              
    \end{longtable}
    
\subsection{Model der Asynchromaschine}
    \includegraphics[scale = 0.6]{images/ModelASM}
    \includegraphics[scale = 0.7]{images/ModelASMZeiger}
    %TODO Kennlinie ASM S27

    \begin{longtable}{| p{.55\textwidth} | p{.45\textwidth}|}
        \hline
        \textbf{Grundgleichungen:}\newline
        \[ \underline{I_1}= \underline{I_{FE}}+\underline{I_\mu}+\underline{I'_2} \approx \underline{I'_2} \]
        \[ \underline{U_1}= R_1 \cdot \underline{I_1}+jX_{\sigma 1}\cdot \underline{I_1}+ \underline{U_h} \approx \underline{U_h} \]
        \textbf{Übersetzungsverhältnis:}\newline
        \[ u=\frac{N_1 \cdot k_{w1}}{N_2 \cdot k_{w2}}\]
        \[ \underline{I'_2}=\underline{I_2 \cdot u} \]
        \[ R'_2 = R_2 \cdot u^2 \]&
        N = Windungszahl \newline
        $ k_w $ = Wicklungsfaktor \newline
        $ R_1, X_{\sigma 1} $= Widerstand und Streureaktanz des Stators \newline
        $ R_2, X_{\sigma 2} $= Widerstand und Streureaktanz des Rotors \newline
        $ R_{fe} $= Eisen-Verlustewiderstand \newline
        $ X_h $= Hauptreaktanz \newline
        $ U_h $= innere Spannung \newline
        $ I_\mu $= Magentisierungsstrom
        \\ \hline      
        
        \textbf{Leerlauf} \newline
        \tabbild[scale=1]{images/ASMLeerlaufZeiger}&
         \newline
        \tabbild[scale=1.3]{images/ASMLeerlauf}
        \\
        
        \[ \underline{I_0}=\underline{I_{FE}} + \underline{I_\mu} \]
        \[ cos(\varphi_0)= \frac{P_0}{U_0 \cdot I_0} \]
        \[ R_{FE}=\frac{U_0}{I_{FE}}=\frac{U_0}{I_0 \cdot cos(\varphi_0)} \]
        \[ X_h = \frac{U_0}{I_\mu}=\frac{U_0}{I_0 \cdot sin(\varphi_0)} \]&
         \textbf{Im Leerlauf wird die Asynchronmaschine an der Welle nicht belastet.} \newline
        \[ R_1 << R_{FE} \qquad X_{\sigma 1} << X_h\]
        \\ \hline
        
        \textbf{Kurzschluss} \newline
        \tabbild[scale=1]{images/ASMKurzschlussZeiger}&
         \newline
        \tabbild[scale=0.8]{images/ASMKurzschluss}
        \\
        
        \[ cos(\varphi_K)= \frac{P_K}{U_K \cdot I_K} \]
        \[ R1 + R'_2 = \frac{U_R}{I_K} = \frac{U_K \cdot cos(\varphi_k)}{I_K} \]
        \[ X_{\sigma 1}+ X'_{\sigma 2}= \frac{U_X}{I_K}=\frac{U_K \cdot sind(\varphi_K)}{I_k} \]
        &
         \textbf{Im Kurzschluss wird der Rotor der Asynchronmaschine blockiert.} \newline
         \[ R_1 << R_{FE} \qquad X_{\sigma 1} << X_h\]
         \\ \hline
         
         \textbf{4Q- Umrichter} \quad FU = Frequenzumrichter\newline
         \tabbild[scale=0.8]{images/4QSchema}&
          \newline
         \tabbild[scale=0.8]{images/4QMn}
         \\ \hline
         
        
        
        
    \end{longtable}
    
    \clearpage
    \pagebreak
