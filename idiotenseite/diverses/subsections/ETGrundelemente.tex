\renewcommand{\arraystretch}{1.5}
\subsection{Grundelemente}
\begin{tabular}{p{1.5cm} p{4.3cm} |p{1.5cm} p{4.3cm}| p{1.5cm} p{4.3cm}}
	\multicolumn{2}{l}{\textbf{Ohmscher Widerstand R}}
	& \multicolumn{2}{l}{\textbf{Kapazitität C}}
	& \multicolumn{2}{l}{\textbf{Induktivität L}} \\
	\multicolumn{2}{l}{$u$ und $i$ können sprunghaft ändern}
	& \multicolumn{2}{l}{$\mathbf{u}$ \textbf{kann nicht sprunghaft ändern}}
	& \multicolumn{2}{l}{$\mathbf{i}$ \textbf{kann nicht sprunghaft ändern}} \\
	 
	\multirow{2}{1.5cm}{
		\includegraphics[width=1.5cm]{./images/zeigerdiag-r.png}}
	& $u(t) = R \cdot i(t)$ 
	& \multirow{2}{1.5cm}{\includegraphics[width=1.5cm]{./images/zeigerdiag-c.png}}
	& $u(t) = \frac{1}{C} \int\limits_0^t i(\tau) d\tau + u(0)$
	& 
	\multirow{2}{1.5cm}{\includegraphics[width=1.5cm]{./images/zeigerdiag-l.png}}
	&$u(t) = L \frac{di(t)}{dt}$\\
	
	&$i(t) = \frac{u(t)}{R}$
	& & $i(t) = C \frac{d u(t)}{dt}$
	& & $i(t) = \frac{1}{L} \int\limits_0^t u(\tau) d\tau + i(0)$\\
	
	& $\underline{Z}_R = R$
	& & $\underline{Z}_C = \frac{1}{j \omega C} = - \frac{j}{\omega C}$
	& & $\underline{Z_L} = j \omega L$\\
	
	\parbox{1.7cm}{\small{nicht linear:}}
	& $R_=(u) = \frac{U}{I(u)}, r_D = \frac{\diff U}{\diff I}\lvert_{U_0}$
	& & $X_C = -\frac{1}{\omega C} \quad B_C = \omega C$
	& & $X_L = \omega L
	\quad B_L = -\frac{1}{\omega L}$ \\
	
	& $P=I^2 \cdot R = \frac{U^2}{R}$
	& & $Q_C= - U^2 \cdot \omega C = - \frac{I^2}{\omega C}$
	& & $Q_L= I^2 \cdot \omega L = \frac{U^2}{\omega L}$\\
	
	& & & $W_C=\frac12 C U_C^2$
	& &$W_L=\frac12 L I_L^2$
\end{tabular}

\subsection{Begriffe der Impedanz und Admittanz}
\begin{tabular}{lllll}
	Scheinwiderstand     &            & $Z = \frac{U_{eff}}{I_{eff}} $                                                                             & $ =
	\sqrt{R^2+X^2}$                                                                                                                                                          & Ohm     \\
	Komplexer Widerstand & Impedanz   & $\underline Z = R + jX = Z \cdot e^{j \varphi}$                                                            & $  = \dfrac{\underline{U}}{\underline{I}} = \dfrac{\underline{U}\cdot\underline{U}^{\ast}}{\underline{S}^*} =  = \dfrac{U^2}{\underline{S}^*} = 
	\dfrac{\underline{S}}{I^2}$ & Ohm     \\
	Komplexer Leitwert   & Admittanz  & $\underline Y = G + jB =
	\frac{1}{\underline Z} = \frac{1}{Z}e^{-j\varphi}$                               & $= \frac{\underline{I}}{\underline{U}}$                                                                                                                                       & Siemens \\
	Wirkwiderstand       & Resistanz  & $R = \Real(\underline Z) $                                                                                 & $ = Z
	\cdot cos(\varphi)$                                                                                                                                                    & Ohm     \\
	Wirkleitwert         & Konduktanz & $G = \Real(\underline Y) $                                                                                 & $ \neq \frac{1}{R}$                                                                                                                                                           & Siemens \\
	Blindwiderstand      & Reaktanz   & $X = \Imag(\underline Z) $                                                                                 & $ = Z
	\cdot sin(\varphi)$                                                                                                                                                    & Ohm     \\
	Blindleitwert        & Suszeptanz & $B = \Imag( \underline Y) $                                                                                & $ \neq \frac{1}{X}$                                                                                                                                                           & Siemens \\
	Phasenverschiebung   &            & $\varphi = \varphi_u - \varphi_i =
	\arctan\left(\frac{\Imag(\underline{Z})}{\Real(\underline{Z})}\right)$ &                                                                                                                                                                               & Radiant
\end{tabular}

% sorry for the mess
\medskip
\begin{tabularx}{\linewidth}{lXl}
	Komplexe Leistung $\underline{S}$ & $\underline{S} = P + jQ = \underline{U} \cdot \underline{I}^* = U \cdot I \cdot e^{j(\varphi_u - \varphi_i)} = \frac{\underline{U^2}}{\underline{Z}^*} = \underline{I^2} \cdot \underline{Z}$ & W   \\
	Scheinleistung $S$                & $S = U \cdot I$ = $\frac{U^2}{Z} = I^2 \cdot Z = \sqrt{P^2 + Q^2}$                                                                                                                            & VA  \\
	Wirkleistung $P$                  & P = $\text{Re}(\underline{S}) = S \cdot \cos(\varphi) = I^2 \cdot \text{Re}(\underline{Z})$                                                                                                   & W   \\
	Blindleistung $Q$                 & $\text{Im}(\underline{S}) = S \cdot sin(\varphi) = P \cdot \tan(\varphi)$                                                                                                                     & Var \\
	Leistungsfaktor $\lambda$         & $\lambda = \frac{P}{S} = \frac{P}{U \cdot I} = cos(\varphi) \text{\tiny (letzeres nur bei sinusförmigen Schwingungen)}$                                                                       &
\end{tabularx}
